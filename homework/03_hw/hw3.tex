\documentclass[]{article}
\usepackage{lmodern}
\usepackage{amssymb,amsmath}
\usepackage{ifxetex,ifluatex}
\usepackage{fixltx2e} % provides \textsubscript
\ifnum 0\ifxetex 1\fi\ifluatex 1\fi=0 % if pdftex
  \usepackage[T1]{fontenc}
  \usepackage[utf8]{inputenc}
\else % if luatex or xelatex
  \ifxetex
    \usepackage{mathspec}
  \else
    \usepackage{fontspec}
  \fi
  \defaultfontfeatures{Ligatures=TeX,Scale=MatchLowercase}
\fi
% use upquote if available, for straight quotes in verbatim environments
\IfFileExists{upquote.sty}{\usepackage{upquote}}{}
% use microtype if available
\IfFileExists{microtype.sty}{%
\usepackage{microtype}
\UseMicrotypeSet[protrusion]{basicmath} % disable protrusion for tt fonts
}{}
\usepackage[margin=1in]{geometry}
\usepackage{hyperref}
\hypersetup{unicode=true,
            pdftitle={Homework 3},
            pdfborder={0 0 0},
            breaklinks=true}
\urlstyle{same}  % don't use monospace font for urls
\usepackage{graphicx,grffile}
\makeatletter
\def\maxwidth{\ifdim\Gin@nat@width>\linewidth\linewidth\else\Gin@nat@width\fi}
\def\maxheight{\ifdim\Gin@nat@height>\textheight\textheight\else\Gin@nat@height\fi}
\makeatother
% Scale images if necessary, so that they will not overflow the page
% margins by default, and it is still possible to overwrite the defaults
% using explicit options in \includegraphics[width, height, ...]{}
\setkeys{Gin}{width=\maxwidth,height=\maxheight,keepaspectratio}
\IfFileExists{parskip.sty}{%
\usepackage{parskip}
}{% else
\setlength{\parindent}{0pt}
\setlength{\parskip}{6pt plus 2pt minus 1pt}
}
\setlength{\emergencystretch}{3em}  % prevent overfull lines
\providecommand{\tightlist}{%
  \setlength{\itemsep}{0pt}\setlength{\parskip}{0pt}}
\setcounter{secnumdepth}{0}
% Redefines (sub)paragraphs to behave more like sections
\ifx\paragraph\undefined\else
\let\oldparagraph\paragraph
\renewcommand{\paragraph}[1]{\oldparagraph{#1}\mbox{}}
\fi
\ifx\subparagraph\undefined\else
\let\oldsubparagraph\subparagraph
\renewcommand{\subparagraph}[1]{\oldsubparagraph{#1}\mbox{}}
\fi
\usepackage{hyperref}
\hypersetup{colorlinks=true,urlcolor=blue}

\title{Homework 3}
\author{}
\date{\vspace{-2.5em}}

\begin{document}
\maketitle

Nothing required to be turned in for question 1. For question 2, submit
your R markdown code as a plain text file with a .Rmd extension. For
question 3, submit your code for breakout exercises 1, 2, and 3 (if
asynchronous) or submit your grade for each student's participation in
your breakout group \textbf{for class on 2020/09/17} (if synchronous) in
a comment at the end of your Rmd file in the following format:

\begin{itemize}
\tightlist
\item
  Susy Student = 1
\item
  Cam Classmate = 1
\item
  Danny Scientist = 0
\end{itemize}

Remember if you are using \texttt{output:\ html\_document}, then the
start of a comment is indicated by \texttt{\textless{}!-\/-} and the end
of comment is indicated by \texttt{-\/-\textgreater{}}.

\begin{enumerate}
\def\labelenumi{\arabic{enumi}.}
\item
  If you do not already have a GitHub account,
  \href{https://github.com/}{sign up for one}.
\item
  Start writing an R Markdown script for your project. The script can
  output whatever format you would like. For now, make sure that your
  script:
\end{enumerate}

\begin{itemize}
\tightlist
\item
  does some sort of data cleaning;
\item
  creates a nice looking table OR a figure (eventually, we'll need
  both);
\item
  includes plain text description of the analysis and/or results.
\end{itemize}

It will benefit you to start thinking about what someone else would need
to do in order to reproduce your script. Next week, we will be putting
this script + data onto GitHub and asking your classmates to download it
and run it themselves.

\begin{enumerate}
\def\labelenumi{\arabic{enumi}.}
\setcounter{enumi}{2}
\tightlist
\item
  If you attended class synchronously, give participation grades to your
  classmates (0 = not present/did not say anything, 1 = present). If you
  attended class asynchronously, include your code for each of the
  breakout exercises from the R Markdown lecture.
\end{enumerate}

\end{document}
